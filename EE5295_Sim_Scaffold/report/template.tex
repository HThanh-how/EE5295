\documentclass[11pt,a4paper]{article}
\usepackage[utf8]{vietnam}
\usepackage{graphicx}
\usepackage{siunitx}
\usepackage{amsmath, amssymb}
\usepackage{booktabs}
\usepackage{hyperref}
\usepackage{caption}
\usepackage{subcaption}

\title{Tiểu luận Phân tích mạch --- EE5295}
\author{Nhóm: \texttt{<điền tên>}}
\date{\today}

\begin{document}
\maketitle

\section{Tổng quan}
\textbf{Chủ đề:} \emph{<điền tên mạch: LDO / SAR ADC / VCO / BGR / OpAmp>}\\
Chức năng, ứng dụng, thông số chính (tóm tắt).

\section{Cơ sở lý thuyết và mô hình}
\subsection{Sơ đồ mạch và nguyên lý}
Mô tả sơ đồ và khối chức năng; đưa phương trình cốt lõi.

\subsection{Phân tích tham số và kỳ vọng}
Tính toán theo lý thuyết: ví dụ với BGR, $V_\text{ref} = V_{BE} + k\Delta V_{BE}$; với VCO, $f(V_\text{ctrl})$; với LDO, line/load regulation, PSRR,...

\section{Kết quả mô phỏng}
\subsection{Thiết lập mô phỏng}
Mô tả công cụ (NGSpice v..., Python), netlist/param, sơ đồ sweep.

\subsection{Kết quả và so sánh với lý thuyết}
Chèn hình/ bảng từ \texttt{data/results.csv}. Ví dụ hình bode RC:
\begin{figure}[h]
\centering
\includegraphics[width=.8\linewidth]{rc_bode_placeholder.pdf}
\caption{Ví dụ minh hoạ bóc tách dữ liệu từ \texttt{rc\_ac.csv}.}
\end{figure}

\subsection{Đánh giá/nhận xét}
Nêu ưu/nhược điểm, sai lệch, giới hạn mô hình.

\section{Điểm cộng: Kết hợp AI (tuỳ chọn)}
Trình bày cách dùng meta-heuristic (ví dụ SA/PSO/SOA) để tối ưu R,C,... nhằm đạt mục tiêu (PSRR, TC, v.v.).

\bibliographystyle{ieeetr}
\bibliography{refs}
\end{document}
