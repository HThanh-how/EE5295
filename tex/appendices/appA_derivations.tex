\chapter{Appendix A: Derivations}
Details on Leeson's equation, noise modeling, and linearization of \(f(V_{ctrl})\).

\section*{A.1 Leeson-style Phase Noise}
Starting from the small-signal phase perturbation model, the single-sideband phase noise around the carrier can be approximated by
\[
  \mathcal{L}(f_\text{off}) \approx 10\log_{10} \Bigg( \frac{F k T}{2 P_{\text{sig}}} \Big[1+\big(\tfrac{f_c}{f_\text{off}}\big)\Big] \Big(\frac{f_0}{2 Q f_\text{off}}\Big)^2 \Bigg) \,\text{dBc/Hz},
\]
where \(F\) is noise factor, \(Q\) effective tank quality factor, \(f_c\) the 1/f corner, and \(P_{\text{sig}}\) signal power at the load. This form motivates our proxy used inside PSO: larger \(Q\) and carrier power, and larger offset reduce \(\mathcal{L}\).

\section*{A.2 Ring Frequency Approximation}
For a ring VCO with \(N\) identical stages of propagation delay \(t_p\),
\[
  f_{osc} \approx \frac{1}{2 N t_p}\,.
\]
With current-starved inverters, the stage delay scales as
\[
  t_p \propto \frac{C_L \; V_{swing}}{I_{ctrl}}\,,\quad \Rightarrow \quad f_{osc}(I_{ctrl}) \approx \frac{I_{ctrl}}{2 N C_L V_{swing}}\,.
\]

\section*{A.3 Linearization of \(f(V_{ctrl})\)}
If the bias generator gives \(I_{ctrl} = g_m \,(V_{ctrl} - V_T)\) locally, then
\[
  f(V_{ctrl}) \approx \frac{g_m}{2 N C_L V_{swing}}\,(V_{ctrl}-V_T)\,\Rightarrow\, K_{VCO} = \frac{\partial f}{\partial V_{ctrl}} \approx \frac{g_m}{2 N C_L V_{swing}}\,.
\]
Nonlinearities (slew-limited edges, device \(g_m\) roll-off) bend the curve; mid-range calibration or bias shaping restores linearity.

\section*{A.4 Time-to-Phase Relation}
For small timing error \(\delta t(t)\), instantaneous phase error is
\[
  \delta\phi(t) \approx 2\pi f_{osc}\, \delta t(t),\quad S_{\phi\phi}(f) = (2\pi f_{osc})^2 S_{tt}(f)\,.
\]
This connects transient jitter post-processing to \(\mathcal{L}(f)\) used in specifications.

\section*{A.5 Example Numbers}
Assume \(f_0=10\,\text{MHz}\), \(Q=20\), \(F=3\), \(P_{sig}=0.5\,\text{mW}\), \(f_c=10\,\text{kHz}\), \(T=300\,\text{K}\). Evaluating the Leeson-style expression gives the following illustrative values:
\begin{table}[H]
  \centering
  \begin{tabular}{lcc}
    \toprule
    Offset & $\mathcal{L}(f)$ & Note \\
    \midrule
    10 kHz & $\approx -85$ dBc/Hz & close-in region \\
    100 kHz & $\approx -100$ dBc/Hz & far from $f_c$ \\
    1 MHz & $\approx -120$ dBc/Hz & roll-off $\propto f_{off}^{-2}$ \\
    \bottomrule
  \end{tabular}
  \caption{Illustrative PN from Leeson-type estimate.}
\end{table}
Integrating \(\mathcal{L}(f)\) over offset bandwidth yields RMS jitter; compare with transient-based extraction for sanity check.


