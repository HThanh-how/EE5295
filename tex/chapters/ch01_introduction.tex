\chapter{Introduction}
\section{Background and Motivation}
Voltage Controlled Oscillators (VCOs) convert a control voltage into an output frequency and are ubiquitous in modern electronic systems. Typical applications include frequency synthesis in Phase-Locked Loops (PLLs), RF transceivers, clock generation and recovery, time-to-digital conversion, and sensor interfaces. As integration scales and performance targets tighten, designers must carefully trade off phase noise, tuning range, power, and silicon area.

\section{Objectives and Scope}
This report aims to provide a practical yet rigorous overview to design, analyze, and integrate VCOs. The scope covers fundamentals of oscillation, architectural choices (ring versus LC), phase-noise modeling, design procedures, PLL integration, measurement methodology, and a case study. The focus is on CMOS-integrated designs while keeping the discussion general enough to apply to other technologies.

\section{Contributions}
The main contributions are:
\begin{itemize}
  \item A structured design workflow from specification to final verification.
  \item Consolidated phase-noise reasoning with actionable reduction techniques.
  \item Comparative guidance between ring and LC VCOs with selection heuristics.
  \item A reproducible case study with tunable parameters and measurement plan.
\end{itemize}

\section{Report Structure}
The report is organized as follows. Chapter~2 reviews oscillation fundamentals, Barkhausen criteria, LC resonance, and quality factor. Chapter~3 surveys VCO architectures with pros/cons. Chapter~4 details ring VCO theory and practical design. Chapter~5 focuses on LC VCO tanks, tuning, and negative resistance topologies. Chapter~6 develops phase-noise models and reduction techniques. Chapter~7 outlines a step-by-step design flow. Chapter~8 integrates VCOs inside PLLs and discusses loop stability and jitter. Chapter~9 covers measurement and calibration. Chapter~10 presents a 10~MHz case study. Chapter~11 concludes and suggests future work. Appendices provide derivations, simulation setups, and datasheets.

\section{Notation and Units}
Frequencies are denoted by \(f\) in Hz, angular frequencies by \(\omega = 2\pi f\) in rad/s. Voltages and currents are expressed in SI units. Phase noise is reported as single-sideband spectral density in dBc/Hz at an offset frequency \(\Delta f\). The VCO gain is \(K_{VCO} = \partial f/\partial V_{ctrl}\) in Hz/V.

\section{Assumptions and Limitations}
Unless stated otherwise, noise sources are treated as small-signal perturbations around the steady-state oscillation. Parasitics are aggregated into effective tank parameters. Nonlinear effects are discussed qualitatively with emphasis on their impact on amplitude regulation and phase noise.

\section{Applications and Target Specifications}
Typical VCO applications include frequency synthesis in PLLs, RF transceivers (LO generation), clock/data recovery, and precision timing for data converters. We target a practical design envelope for this report.

\begin{table}[H]
  \centering
  \begin{tabular}{lll}
    \toprule
    Parameter & Target & Rationale \\
    \midrule
    Center frequency $f_0$ & 10 MHz & easy lab instrumentation \\
    Tuning range & $\pm20\%$ & cover PVT and app variations \\
    $K_{VCO}$ & $\sim 1$–$3$ MHz/V & PLL bandwidth, spur control \\
    Phase noise @100 kHz & $< -100$ dBc/Hz & low-jitter requirement \\
    Supply $V_{DD}$ & 1.8 V & common low-voltage node \\
    Power & $< 10$ mW & portable/embedded use \\
    Area (LC) & $\le 0.3\,\mathrm{mm}^2$ & on-chip feasibility \\
    \bottomrule
  \end{tabular}
  \caption{Target specification summary used throughout the report}
\end{table}

