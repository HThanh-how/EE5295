\chapter{LC VCO}
\section{Bể cộng hưởng và tần số}
Tần số xấp xỉ: \( f \approx \frac{1}{2\pi\sqrt{LC}} \). Điều chỉnh bằng varactor \(C(V_{ctrl})\).
\section{Độ lợi \(K_{VCO}\) và tuyến tính}
Biểu thức tuyến tính cục bộ quanh \(V_{mid}\): \( f(V)=f_0+K_{VCO}(V-V_{mid}) \). Thiết kế varactor song song/đối xứng để cải thiện tuyến tính.
\section{Nhiễu pha và hệ số \(Q\)}
Theo Leeson \cite{leeson1966}: \(\mathcal{L}(\Delta f) \propto f_0^2/(Q^2\Delta f^2)\). Tăng \(Q\) bằng cuộn cảm chất lượng cao, giảm tổn hao, lọc dòng đuôi.
\section{Kiến trúc mạch}
Cross-coupled NMOS/PMOS, tail filtering, switched-cap tuning coarse/fine. Cân bằng biên độ để tránh kéo méo.
\section{Illustrative Schematic}
\begin{figure}[H]
  \centering
  \begin{circuitikz}[european]
    \draw (0,0) node[ground]{} to[I, l=$I_{bias}$] (0,3) -- (2,3)
          (2,3) to[L,l=$L$] (4,3) -- (6,3)
          (4,3) to[C,l=$C(V_{ctrl})$] (4,0)
          (6,3) to[open,v^>=$V_{out}$] (6,0);
  \end{circuitikz}
  \caption{LC VCO with varactor control}
\end{figure}
\section{Khuyến nghị thiết kế}
Chọn dải varactor vừa đủ, thêm bank tụ chuyển mạch cho coarse tuning, cách ly nguồn, và dùng buffer xuất công suất thấp.


