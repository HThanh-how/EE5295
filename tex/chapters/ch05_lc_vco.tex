\chapter{LC VCO}
\section{Resonant Tank and Frequency}
The oscillation frequency is approximately
\[
 f \approx \frac{1}{2\pi\sqrt{LC}}
\]
and is tuned by a voltage-dependent capacitance \(C(V_{ctrl})\) using varactors.
\section{VCO Gain \(K_{VCO}\) and Linearity}
Locally around \(V_{mid}\), the tuning law is
\[
 f(V)=f_0+K_{VCO}(V-V_{mid})
\]
Parallel/symmetric varactor arrangements improve linearity and reduce AM-to-PM conversion.
\section{Phase Noise and Quality Factor}
Following Leeson \cite{leeson1966}, phase noise scales roughly as \(\mathcal{L}(\Delta f) \propto f_0^2/(Q^2\Delta f^2)\). Improving \(Q\) via high-quality inductors and reduced loss, and filtering tail current, reduces phase noise.
\section{Circuit Topologies}
Cross-coupled NMOS/PMOS cores, tail filtering, and switched-capacitor coarse/fine tuning are common. Amplitude balancing avoids core overdrive.
\section{Illustrative Schematic}
\begin{figure}[H]
  \centering
  \begin{circuitikz}[european]
    \draw (0,0) node[ground]{} to[I, l=$I_{bias}$] (0,3) -- (2,3)
          (2,3) to[L,l=$L$] (4,3) -- (6,3)
          (4,3) to[C,l=$C(V_{ctrl})$] (4,0)
          (6,3) to[open,v^>=$V_{out}$] (6,0);
  \end{circuitikz}
  \caption{LC VCO with varactor control}
\end{figure}
\section{Design Guidelines}
Select a just-enough varactor range, add switched-capacitor banks for coarse tuning, isolate supplies, and use a low-load output buffer.


