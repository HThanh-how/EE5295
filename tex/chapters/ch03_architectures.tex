\chapter{VCO Architectures}
\section{Taxonomy}
Common categories include ring VCOs, LC VCOs, and relaxation VCOs. Selection depends on target frequency, phase-noise budget, tuning range, power, and silicon area.
\begin{figure}[H]
  \centering
  \begin{tikzpicture}[node distance=1.6cm, >=Latex]
    \node[draw, rounded corners, minimum width=3.2cm, minimum height=0.9cm, fill=gray!10] (vco) {VCO};
    \node[draw, rounded corners, below left=of vco, minimum width=3.2cm] (ring) {Ring VCO};
    \node[draw, rounded corners, below=of vco, minimum width=3.2cm] (lc) {LC VCO};
    \node[draw, rounded corners, below right=of vco, minimum width=3.2cm] (relax) {Relaxation VCO};
    \draw[->] (vco) -- (ring);
    \draw[->] (vco) -- (lc);
    \draw[->] (vco) -- (relax);
  \end{tikzpicture}
  \caption{Taxonomy of common VCO architectures}
\end{figure}
\section{Ring VCO}
Pros: compact, wide tuning, easy integration; Cons: higher phase noise, supply sensitivity. Frequency roughly \(1/(2Nt_p)\).
\subsection*{Delay and Bias Control}
For a current-controlled ring, the per-stage delay scales as
\[
 t_p \propto \frac{C_L\,V_{swing}}{I_{bias}}
\]
leading to
\[
 f_{osc} \approx \frac{1}{2 N t_p} \propto \frac{I_{bias}}{C_L\,V_{swing}}
\]
Linearization uses bias shapers to map \(V_{ctrl}\) to \(I_{bias}\) linearly over the useful range.
\subsection*{Supply and Substrate Noise}
Use local LDOs or filters for bias, star-grounding, guard rings, and differential structures to mitigate AM-to-PM conversion.
\subsection*{Digitally Controlled Oscillator (DCO)}
Replace the analog control with a digitally switched load or current bank. Coarse/fine segmentation provides monotonic tuning with calibration.
\section{LC VCO}
Pros: low phase noise, good spectral purity; Cons: larger area (inductors), narrower tuning. Frequency set by \(1/(2\pi\sqrt{LC})\).
\subsection*{Tank and Negative Resistance}
The tank resonance is
\[
 f_0 = \frac{1}{2\pi\sqrt{LC}}
\]
Active devices provide an effective negative resistance \(-R_n\) to cancel the parallel loss \(R_p\). Startup requires \(|R_n|>R_p\).
\subsection*{Varactors and Tuning Law}
MOS varactors (accumulation) and PN junction varactors trade off \(Q\), tuning range, and voltage headroom. The local linear law around \(V_{mid}\) is
\[
 f(V_{ctrl}) \approx f_0 + K_{VCO}(V_{ctrl}-V_{mid})
\]
Use switched-cap banks for coarse tuning and varactor for fine tuning to ensure overlap across PVT.
\subsection*{Phase Noise Considerations}
Leeson-like behavior motivates maximizing tank \(Q\), minimizing device noise factor, and filtering the tail current. Symmetric differential cores reduce even-order distortion.
\section{Relaxation VCO}
Pros: very wide tuning, simple concept; Cons: poor spectral purity for RF. Suitable for low-to-mid frequency timers.
\subsection*{Core Principle}
Integrate a current on a capacitor until a threshold, then reset via a comparator. Frequency is approximately
\[
 f \approx \frac{I}{C\,\Delta V}
\]
where \(\Delta V\) is the hysteresis window.
\subsection*{Use Cases}
Good for timers, low-frequency modulators, or as coarse DCO elements inside synthesizers.
\section{Comparison Guidance}
When phase noise is critical (RF synthesizers), choose LC. When area and wide range dominate (digitally controlled oscillators, on-chip clocks), choose ring. Hybrid strategies combine coarse digital banks with fine analog tuning.

\begin{table}[H]
  \centering
  \begin{tabular}{llll}
    \toprule
    Architecture & Phase Noise & Tuning Range & Area/Complexity \\
    \midrule
    Ring & Higher & Wide & Small / Low \\
    LC & Lower & Moderate & Larger / Medium \\
    Relaxation & Poor (RF) & Very Wide & Small / Low \\
    \bottomrule
  \end{tabular}
  \caption{High-level comparison of VCO architectures}
\end{table}

\section{Quantitative Phase-Noise Comparison}
For a fair comparison at fixed \(f_0\) and power, a simplified Leeson-inspired view suggests
\[
 \mathcal{L}(\Delta f) \propto \frac{F k T}{2 P_s}\,\frac{f_0^2}{Q^2\,\Delta f^2}
\]
Thus LC designs (higher \(Q\)) generally outperform ring designs if power is similar. Improving ring phase noise requires either more power or architectural tricks (e.g., injection locking, multiphase averaging).

\subsection*{Relative PN map: LC vs Ring}
Let subscripts R/L denote Ring/LC. Under similar measurement conditions (same \(f_0,\,\Delta f,\,T\)):
\[
 \frac{\mathcal{L}_\mathrm{LC}}{\mathcal{L}_\mathrm{Ring}} \approx \frac{F_\mathrm{LC}}{F_\mathrm{Ring}}\,\Big(\frac{Q_\mathrm{Ring}}{Q_\mathrm{LC}}\Big)^2\,\Big(\frac{P_\mathrm{Ring}}{P_\mathrm{LC}}\Big)
\]
Hence, \(\mathcal{L}_\mathrm{LC}<\mathcal{L}_\mathrm{Ring}\) when
\[
 \Big(\frac{Q_\mathrm{LC}}{Q_\mathrm{Ring}}\Big)^2 > \frac{F_\mathrm{LC}}{F_\mathrm{Ring}}\,\Big(\frac{P_\mathrm{Ring}}{P_\mathrm{LC}}\Big)
\]
This inequality provides a quick design map “when LC > Ring” given target \(Q\) and power ratios.

\section{KVCO Linearity and DCO INL/DNL}
The local linear tuning law is
\[
 f(V_{ctrl}) \approx f_0 + K_{VCO}(V_{ctrl}-V_{mid})
\]
Define linearity error as \(\epsilon_{lin} = \max\_i |f_i - \hat{f}_i|/\text{FS}\), where \(\hat{f}_i\) is a linear fit and FS is the full-scale span. For a DCO, define code density DNL/INL on the frequency codes; thermometer coding for fine segments reduces DNL spikes.

\subsection*{RMS jitter from phase-noise mask}
Given a single-sideband mask \(\mathcal{L}(f)\) and carrier \(f_0\), the RMS jitter over offsets \([f_1, f_2]\) can be estimated by
\[
 \sigma_t \approx \frac{1}{2\pi f_0}\, \sqrt{ 2\int\limits_{f_1}^{f_2} \mathcal{L}(f)\,df }
\]
For tabulated points \(\{f_k,\,\mathcal{L}_k\}\), use a trapezoidal sum in linear power units.

\section{Spur Mechanisms and Mitigations}
Spurs arise from switched tuning elements, supply ripple, and CP/LF ripple inside PLLs. Mitigations include scrambling, dynamic element matching, spread-spectrum banks, and better decoupling/regulation.

\subsection*{PLL CP ripple to VCO spur}
Charge-pump ripple at the comparison frequency leaks through the loop filter to the VCO control node, creating reference spurs. Remedies: increase attenuation at \(f_{ref}\), use active filters, or add feed-forward cancellation. Scrambling the coarse bank reduces deterministic tones.

\section{DCO Coarse/Fine Architecture}
\begin{figure}[H]
  \centering
  \begin{tikzpicture}[node distance=1.4cm, >=Latex]
    \node[draw, rounded corners, minimum width=2.6cm] (coarse) {Coarse Bank (Binary)};
    \node[draw, rounded corners, right=of coarse, minimum width=2.6cm] (fine) {Fine Bank (Thermo)};
    \node[draw, rounded corners, right=of fine, minimum width=2.6cm] (core) {VCO Core};
    \draw[->] (coarse) -- (fine);
    \draw[->] (fine) -- (core);
  \end{tikzpicture}
  \caption{Digitally Controlled Oscillator with coarse/fine segmented banks}
\end{figure}

\section{Example KVCO Curve}
\begin{figure}[H]
  \centering
  \begin{tikzpicture}[scale=1]
    \draw[->] (-0.2,0) -- (7,0) node[right] {$V_{ctrl}$};
    \draw[->] (0,-0.2) -- (0,4) node[above] {$f$};
    \draw[blue, thick] (0.5,0.6) .. controls (2,1.1) and (4,2.5) .. (6,3.6);
    \draw[red, dashed] (0.5,0.7) -- (6,3.4);
    \node[blue] at (4.9,3.2) {actual};
    \node[red] at (2.6,2.3) {linear fit};
  \end{tikzpicture}
  \caption{KVCO linearization: actual curve vs local linear fit}
\end{figure}

\section{PN Targets and Jitter Table}
\begin{table}[H]
  \centering
  \begin{tabular}{llll}
    \toprule
    Offset & Target PN (dBc/Hz) & Integrated band & Comment \\
    \midrule
    10 kHz  & $<-80$  & 1–20 kHz  & close-in flicker corner \\
    100 kHz & $<-100$ & 20–200 kHz & main loop bandwidth region \\
    1 MHz   & $<-120$ & 200 k–2 M & far-out thermal floor \\
    \bottomrule
  \end{tabular}
  \caption{Example phase-noise targets and integration bands}
\end{table}

\section{Coarse/Fine Example and PVT Overlap}
Consider a DCO with 5-bit coarse (binary) and 6-bit fine (thermometer) banks. Let coarse LSB be \(\Delta f_c\) and fine span be \(\approx 1.5\,\Delta f_c\). Then any adjacent coarse steps are bridged by fine codes to guarantee overlap across PVT (\(\pm10\%\,VDD\), TT/SS/FF, \(-40\,\text{to}\,125^\circ\!\text{C}\)). A calibration loop trims the operating code toward the closest target frequency while dithering below spur limits.

\section{Startup Margin and PVT Sweep}
\begin{table}[H]
  \centering
  \begin{tabular}{lll}
    \toprule
    Condition & Startup margin $M_{su}$ & Note \\
    \midrule
    TT, $V_{DD}$, 25$^\circ$C & $\times 2.5$ & nominal target \\
    SS, $-10\%\,V_{DD}$, 125$^\circ$C & $\times 3.0$ & hardest corner \\
    FF, $+10\%\,V_{DD}$, $-40^\circ$C & $\times 2.0$ & fast/high-supply \\
    \bottomrule
  \end{tabular}
  \caption{Recommended startup margins across PVT corners}
\end{table}

\section{Layout and Power Delivery Checklist}
\begin{itemize}
  \item Guard rings around VCO core; use triple-well isolation where available.
  \item Differential routing with symmetry; minimize loop area; shield sensitive nodes.
  \item Multi-band decoupling (MIM + MOS + deep trench if available) near bias nodes.
  \item Separate analog/digital ground; star-point returns; dedicated LDO for bias.
  \item Keep switched banks away from high-impedance tank nodes; use common-centroid arrays for varactors.
\end{itemize}

\section{Quantitative Architecture Table at same $f_0$}
\begin{table}[H]
  \centering
  \begin{tabular}{lllll}
    \toprule
    Architecture & PN@100 kHz & Range (\%) & Power (mW) & Area (mm$^2$) \\
    \midrule
    Ring & $-95$ & $\ge 50$ & 2 & $\le 0.05$ \\
    LC   & $-110$ & 20–30 & 3 & 0.15–0.30 \\
    Relaxation & $-80$ & $\ge 80$ & 1 & $\le 0.03$ \\
    \bottomrule
  \end{tabular}
  \caption{Illustrative quantitative comparison at $f_0$ = 100 MHz}
\end{table}

\section{Calibration and Linearity Strategies}
\subsection*{Coarse/Fine Tuning}
Combine a binary-weighted coarse bank (range coverage) with a thermometer-coded fine bank (monotonicity). Overlay analog fine control to smooth residual steps.
\subsection*{Background Calibration}
Track PVT drift by periodically dithering around the operating code and minimizing a phase/frequency error metric. Keep dither below spur tolerance.
\subsection*{AM-to-PM Mitigation}
Stabilize swing (e.g., amplitude regulation loops) to reduce AM-to-PM conversion and improve phase-noise-to-jitter translation.

\section{Design Checklists}
\begin{itemize}
  \item Specify phase-noise mask, tuning range, power, and spur limits.
  \item Choose architecture; for LC, pick \(L\) and varactor ranges to cover PVT.
  \item Ensure startup margin and verify across corners and temperatures.
  \item Plan coarse/fine segmentation and calibration hooks.
  \item Simulate PSS/Pnoise (or time-domain jitter) and iterate sizing.
  \item Floorplan: route differential, shield sensitive nodes, decouple supplies.
\end{itemize}


