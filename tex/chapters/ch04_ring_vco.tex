\chapter{Ring VCO}
\section{General Structure}
A ring VCO consists of \(N\) inverting stages connected in a loop so that the total phase shift is \((2k+1)\pi\). The fundamental frequency is approximately
\[
f_{osc} \approx \frac{1}{2 N t_{p}(I_{bias}, V_{ctrl})}
\]
where \(t_p\) is the per-stage propagation delay controlled by bias current/voltage.

\section{Delay Model and \(K\_{VCO}}\)
With current control \(I_{ctrl}\), one has
\[
t_p \propto \frac{C_L\,V_{swing}}{I_{ctrl}}
\]
thus
\[
f_{osc}(I_{ctrl}) \approx \frac{I_{ctrl}}{2 N C_L V_{swing}}\,,\quad K_{VCO} = \frac{\partial f}{\partial V_{ctrl}} = \frac{\partial f}{\partial I_{ctrl}}\,\frac{\partial I_{ctrl}}{\partial V_{ctrl}}
\]
Design chooses \(K_{VCO}\) large enough for acquisition margin but not excessively sensitive to noise. For a transconductor that maps \(V_{ctrl}\) to current, e.g., \(I_{ctrl} = g_m (V_{ctrl}-V_T)\) within range, one gets \(K_{VCO} \approx \tfrac{g_m}{2 N C_L V_{swing}}\).

\section{Current-Starved Inverter Topology}
The current-starved inverter limits the charge/discharge current via a bias device, making delay tunable by bias.
\begin{figure}[H]
  \centering
  \begin{tikzpicture}[scale=1]
    \node[draw, minimum width=3.2cm, minimum height=1cm] (stage) at (0,0) {Current-starved INV};
    \node[right=1.2cm of stage] (dots) {$\cdots$};
    \node[draw, minimum width=3.2cm, minimum height=1cm, right=2.4cm of dots] (stageN) {Stage $N$};
    \draw[-Latex] (stage) -- (dots);
    \draw[-Latex] (dots) -- (stageN);
    \draw[-Latex] (stageN.north) to[out=45,in=135] ++(1.2,0) to[out=-45,in=45] (stageN.south);
  \end{tikzpicture}
  \caption{Conceptual chain of current-starved inverters}
\end{figure}

\section{Small-Signal Start-Up and Amplitude Regulation}
The loop must have net gain $>1$ at startup; as the swing grows, device nonlinearity (effective resistance increase) reduces loop gain to unity. Slew-limited edges and saturation define \(V_{swing}\) and impact \(t_p\).

\section{Phase Noise and Jitter}
Ring VCOs lack a high-\(Q\) tank, so phase noise is typically worse than LC. A rough guideline relates time uncertainty per stage \(\sigma_{t,stg}\) to output jitter:
\[
 \sigma_{t,\,out} \approx \sqrt{N}\,\sigma_{t,stg}
\]
Mitigations: increase bias (faster edges), optimize device sizes, isolate supplies, differential routing, and multiphase averaging. Injection locking can further clean phase noise around the carrier at the cost of lock range.

\subsection*{Injection Locking (Adler's Equation)}
With a periodic injection at frequency \(\omega_{inj}\) near \(\omega_0\), the phase dynamics follow Adler's equation \cite{adler1946}:
\[
 \frac{d\phi}{dt} = \Delta\omega - \Delta\omega_L \sin\phi
\]
where \(\Delta\omega = \omega_0 - \omega_{inj}\) and \(\Delta\omega_L\) is the lock range, proportional to injection strength and inversely to oscillator amplitude. Within lock, the oscillator inherits the injector's spectral purity near the carrier, reducing close-in phase noise.

\section{Linearizing \(f\)–\(V_{ctrl}\)}
Techniques include: (i) bias shapers to linearize \(I_{ctrl}(V_{ctrl})\); (ii) cascoding to stabilize swing; (iii) segmented digital calibration to correct residual curvature; (iv) temperature compensation in the bias reference.

\section{Numerical Design Example}
Target \(f_{osc}=\SI{500}{\mega\hertz}\) with \(N=5\), \(V_{swing}=\SI{0.6}{V}\), and \(C_L=\SI{15}{fF}\). To set midrange frequency:
\[
 I_{ctrl,mid} \approx 2 N C_L V_{swing} f_{osc} = 2\cdot 5\cdot 15\times10^{-15}\cdot 0.6\cdot 5\times10^{8} \approx \SI{45}{\micro A}
\]
If \(g_m=\SI{200}{\micro A/V}\), then \(K_{VCO}\approx \tfrac{g_m}{2 N C_L V_{swing}} \approx \SI{2.2}{\mega Hz/V}\). Choose bias range such that coarse/fine calibration covers PVT drift.

\subsection*{Stage Sizing Trade-offs}
Larger devices reduce thermal noise but increase load \(C_L\) and power. There is an optimum where the product of edge rate and noise spectral density minimizes jitter. Simulate a sweep of widths and extract RMS jitter to locate the optimum.

\section{Supply and Substrate Noise}
Use local LDO/filters, guard rings, triple-well isolation, star-grounding, and symmetric routing. Keep switched banks away from high-impedance nodes to reduce AM-to-PM conversion.

\section{Design Targets and Checklist}
\begin{table}[H]
  \centering
  \begin{tabular}{lll}
    \toprule
    Item & Target & Note \\
    \midrule
    $f$ range & $\pm 30\%$ & includes PVT margin \\
    $K_{VCO}$ & $1$–$5$ MHz/V & linear around mid-range \\
    PN@100 kHz & $< -95$ dBc/Hz & bias/isolation dependent \\
    Power & $< 3$ mW & at $V_{DD}$ specified \\
    Startup & $\times 2$–$3$ & across corners \\
    \bottomrule
  \end{tabular}
  \caption{Example design targets for a ring VCO}
\end{table}

\begin{itemize}
  \item Ensure startup margin across TT/SS/FF, $\pm10\%\,V_{DD}$, $-40$ to $125^\circ$C.
  \item Verify monotonic tuning; add background calibration hooks.
  \item Simulate time-domain jitter and supply sensitivity (PSRR).
  \item Layout symmetry, short return paths, and decoupling close to bias nodes.
  \item Consider injection locking or multiphase averaging if the phase-noise mask is aggressive.
\end{itemize}

\section{Three-Stage Example}
\begin{figure}[H]
  \centering
  \begin{tikzpicture}[scale=1]
    \node[draw, minimum width=2.2cm, minimum height=1cm] (inv1) at (0,0) {INV1};
    \node[draw, minimum width=2.2cm, minimum height=1cm] (inv2) at (4,0) {INV2};
    \node[draw, minimum width=2.2cm, minimum height=1cm] (inv3) at (8,0) {INV3};
    \draw[-latex] (inv1) -- (inv2);
    \draw[-latex] (inv2) -- (inv3);
    \draw[-latex] (8,0.5) to[out=45,in=135] (10,0) to[out=-45,in=45] (8,-0.5);
    \draw[-latex] (8,-0.5) to[out=-135,in=-45] (0,-0.5) to[out=135,in=-135] (0,0.5);
    \node at (10.6,0) {$V_{out}$};
  \end{tikzpicture}
  \caption{Three-stage ring VCO with feedback loop}
\end{figure}

\section{Practical Design Guidelines}
Use odd \(N\), buffer the output, provide a clean programmable bias, keep symmetric layout, and separate analog/digital grounds. Consider multiphase outputs for divider/PLL interfaces to average jitter.


